%% LyX 1.3 created this file.  For more info, see http://www.lyx.org/.
%% Do not edit unless you really know what you are doing.
\documentclass[english, 12pt]{article}
\usepackage{times}
%\usepackage{algorithm2e}
\usepackage{url}
\usepackage{bbm}
\usepackage[T1]{fontenc}
\usepackage[latin1]{inputenc}
\usepackage{geometry}
\geometry{verbose,letterpaper,tmargin=2cm,bmargin=2cm,lmargin=1.5cm,rmargin=1.5cm}
\usepackage{rotating}
\usepackage{color}
\usepackage{graphicx}
\usepackage{amsmath, amsthm, amssymb}
\usepackage{setspace}
\usepackage{lineno}
\usepackage{hyperref}
\usepackage{bbm}
\usepackage{makecell}

\renewcommand{\arraystretch}{1.3}

\usepackage{xr}
\externaldocument{paper-misspec-supp}

%\linenumbers
%\doublespacing
\onehalfspacing
%\usepackage[authoryear]{natbib}
\usepackage{natbib} \bibpunct{(}{)}{;}{author-year}{}{,}

%Pour les rajouts
\usepackage{color}
\definecolor{trustcolor}{rgb}{0,0,1}

\usepackage{dsfont}
\usepackage[warn]{textcomp}
\usepackage{adjustbox}
\usepackage{multirow}
\usepackage{graphicx}
\graphicspath{{../figures/}}
\DeclareMathOperator*{\argmin}{\arg\!\min}
\usepackage{algorithm}
\usepackage{algpseudocode}

\let\tabbeg\tabular
\let\tabend\endtabular
\renewenvironment{tabular}{\begin{adjustbox}{max width=0.9\textwidth}\tabbeg}{\tabend\end{adjustbox}}

\makeatletter

%%%%%%%%%%%%%%%%%%%%%%%%%%%%%% LyX specific LaTeX commands.
%% Bold symbol macro for standard LaTeX users
%\newcommand{\boldsymbol}[1]{\mbox{\boldmath $#1$}}

%% Because html converters don't know tabularnewline
\providecommand{\tabularnewline}{\\}

\usepackage{babel}
\makeatother


\begin{document}


\title{Overcoming multiple sources of misspecification\\in GWAS summary statistics}
\author{Florian Priv\'e,$^{\text{1,}*}$ Timothy S. H. Mak,$^{\text{2}}$ and Bjarni J. Vilhj\'almsson$^{\text{1,3}}$}

\date{~ }
\maketitle

\noindent$^{\text{\sf 1}}$National Centre for Register-Based Research, Aarhus University, Aarhus, 8210, Denmark. \\
\noindent$^{\text{\sf 2}}$Fano Labs, Hong Kong \\
\noindent$^{\text{\sf 3}}$Bioinformatics Research Centre, Aarhus University, Aarhus, 8000, Denmark. \\
\noindent$^\ast$To whom correspondence should be addressed.\\

\noindent Contact: \url{florian.prive.21@gmail.com}

\vspace*{8em}

\abstract{
}

%%%%%%%%%%%%%%%%%%%%%%%%%%%%%%%%%%%%%%%%%%%%%%%%%%%%%%%%%%%%%%%%%%%%%%%%%%%%%%%%

\clearpage

%%%%%%%%%%%%%%%%%%%%%%%%%%%%%%%%%%%%%%%%%%%%%%%%%%%%%%%%%%%%%%%%%%%%%%%%%%%%%%%%

\section*{Introduction}

%%%%%%%%%%%%%%%%%%%%%%%%%%%%%%%%%%%%%%%%%%%%%%%%%%%%%%%%%%%%%%%%%%%%%%%%%%%%%%%%

\section*{Results}

\subsection{Misspecification of GWAS sample sizes}


%%%%%%%%%%%%%%%%%%%%%%%%%%%%%%%%%%%%%%%%%%%%%%%%%%%%%%%%%%%%%%%%%%%%%%%%%%%%%%%%

\section*{Discussion}


%%%%%%%%%%%%%%%%%%%%%%%%%%%%%%%%%%%%%%%%%%%%%%%%%%%%%%%%%%%%%%%%%%%%%%%%%%%%%%%%


\section*{Materials and Methods}

\subsection*{New implementation of lassosum in bigsnpr}

Instead of using a regularized version of the correlation matrix $R$ parameterized by $s$, $R_s = (1 - s) R + s I$, we use $R_{\delta} = R + \delta I$, which makes it clearer that lassosum is also using L2-regularization (therefore elastic-net). 
Then, from \cite{mak2017polygenic}, the solution from lassosum can be obtained by iteratively updating 
\[
\beta_j^{(t)} =
\begin{cases}
\text{sign}\left(u_j^{(t)}\right) \left(\left|u_j^{(t)}\right| - \lambda\right) / \left(\widetilde{X}_j^T \widetilde{X}_j + \delta\right) & \text{if } \left|u_j^{(t)}\right| > \lambda ~, \\
0 & \text{otherwise.}
\end{cases}
\]
where 
\[
u_j^{(t)} = r_j - \widetilde{X}_j^T \left( \widetilde{X} \beta^{(t-1)} - \widetilde{X}_j \beta_j^{(t-1)} \right) ~.
\]
Following notations from \cite{prive2020ldpred2} and denoting $\widetilde{X} = \frac{1}{\sqrt{n-1}} C_n G S^{-1}$, where $G$ is the genotype matrix, $C_n$ is the centering matrix and $S$ is the diagonal matrix of standard deviations of the columns of $G$.
Then $\widetilde{X}_j^T \widetilde{X} = R_{j,.} = R_{.,j}^T$, $\widetilde{X}_j^T \widetilde{X}_j = 1$ and
\[
u_j^{(t)} = \beta_j^{(t-1)} + \widehat{\beta}_j - R_{.,j}^T \beta^{(t-1)} ~,
\]
where $r_j = \widehat{\beta}_j =  \dfrac{\widehat{\gamma}_j}{\sqrt{n_j ~ \text{se}(\widehat{\gamma}_j)^2 + \widehat{\gamma}_j^2}}$ and $\widehat{\gamma}_j$ is the GWAS effect of variant $j$ and $n$ is the GWAS sample size \cite[]{mak2017polygenic,prive2021high}.
Then the most time-consuming part is computing $R_{.,j}^T \beta^{(t-1)}$.
To make this faster, instead of computing $R_{.,j}^T \beta^{(t-1)}$ at each iteration ($j$ and $t$), we can start with a vector with only 0s initially (for all $j$) since $\beta^{(0)} \equiv 0$, and then updating this vector when $\beta_j^{(t)} \neq \beta_j^{(t-1)}$ only. Note that only positions $k$ for which $R_{k,j} \neq 0$ must be updated in this vector $R_{.,j}^T \beta^{(t-1)}$. 

In this new implementation of the lassosum model, the input parameters are the correlation matrix $R$, the GWAS summary statistics ($\widehat{\gamma}_j$, $\text{se}(\widehat{\gamma}_j)$ and $n_j$), and the two hyper-parameters $\lambda$ and $\delta$. 
Therefore, except for the hyper-parameters, this is the exact same input as for LDpred2 \cite[]{prive2020ldpred2}.
We try $\delta \in \{0.001, 0.005, 0.02, 0.1, 0.6, 3\}$ by default in lassosum2, instead of $s \in \{0.2, 0.5, 0.8, 1.0\}$ in lassosum.
For $\lambda$, the default in lassosum uses a sequence of 20 values equally spaced on a log scale between 0.1 and 0.001; we now use a sequence between $\lambda_0$ and $\lambda_0 / 100$ by default in lassosum2, where $\lambda_0 = \max_j \left|\widehat{\beta}_j\right|$ is the minimum $\lambda$ for which no variable enter the model because the L1-regularization is too strong.
Note that we do not provide an ``auto'' version using pseudo-validation (as in \cite{mak2017polygenic}) as we have not found it to be very robust (Figure \ref{fig:pseudoval} [TODO: REGENERATE THIS FIG AND CODE]).
Also note that, as in LDpred2, we run lassosum2 genome-wide using a sparse correlation matrix which assumes that variants further away than 3 cM are not correlated, and therefore we do not require splitting the genome into independent LD blocks anymore (as is required in lassosum).


%%%%%%%%%%%%%%%%%%%%%%%%%%%%%%%%%%%%%%%%%%%%%%%%%%%%%%%%%%%%%%%%%%%%%%%%%%%%%%%%

%\clearpage
%\vspace*{5em}

\section*{Code and results availability}

All code used for this paper is available at %\url{https://github.com/privefl/paper-lassosum2/tree/master/code}.
Latest versions of R package bigsnpr can be installed from GitHub.
A tutorial on running lassosum2 along with LDpred2 using R package bigsnpr is available at \url{https://privefl.github.io/bigsnpr-extdoc/polygenic-scores-pgs.html}.
We have extensively used R packages bigstatsr and bigsnpr \cite[]{prive2017efficient} for analyzing large genetic data, packages from the future framework \cite[]{bengtsson2020unifying} for easy scheduling and parallelization of analyses on the HPC cluster, and packages from the tidyverse suite \cite[]{wickham2019welcome} for shaping and visualizing results.

\section*{Acknowledgements}

Authors thank GenomeDK and Aarhus University for providing computational resources and support that contributed to these research results.
This research has been conducted using the UK Biobank Resource under Application Number 58024.

\section*{Funding}

F.P.\ and B.J.V.\ are supported by the Danish National Research Foundation (Niels Bohr Professorship to Prof. John McGrath).
B.J.V.\ is also supported by a Lundbeck Foundation Fellowship (R335-2019-2339).

\section*{Declaration of Interests}

The authors declare no competing interests.

%%%%%%%%%%%%%%%%%%%%%%%%%%%%%%%%%%%%%%%%%%%%%%%%%%%%%%%%%%%%%%%%%%%%%%%%%%%%%%%%

%\clearpage

\bibliographystyle{natbib}
\bibliography{refs}

%%%%%%%%%%%%%%%%%%%%%%%%%%%%%%%%%%%%%%%%%%%%%%%%%%%%%%%%%%%%%%%%%%%%%%%%%%%%%%%%


\end{document}
